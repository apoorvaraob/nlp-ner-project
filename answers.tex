\documentclass[twoside]{article}
% \usepackage{courier}
% \usepackage{tikz}
% \usepackage{float}
\usepackage{listings}
% \usepackage{color}
% \usetikzlibrary{arrows,automata}
% \usepackage{listing}
\title{CS 585: Project Milestone 1}
\author{Armand Halbert and Aproova Rao Balevalachilu}
\begin{document}
\maketitle
\begin{enumerate}
    \item 
We found that our classifier likes to guess O most of the time, as most items are non-named entities. Deciding what is a named entity is hard. We had trouble ourselves deciding what constituted a named entity.

\begin{lstlisting}
    
Output of tageval:
F = 0.0826,  Prec = 0.9032 (28/31),  Rec = 0.0433 (28/647)
(1000 sentences, 19378 tokens, 647 gold spans, 31 predicted spans)
\end{lstlisting}
\newline
\newline
\item We used capital letters as a feature as whether or not it's a feature vector. This causes problems for words like Prison or Day. As for false negatives, Daily Mail is one example. Since it is seen as separate words, each word is a non-named entity.
But together, they are a newspaper. So dealing with I terms needs to be improved. Chunking is one way to do so.
\newline
\newline
\item From our feature set, we found that the model emphasizes words with capital letter tend to be given the highest weight. Symbols such as ! are given a larger weight. If an item does not have digits in it, the model is more likely to classify it as O(Non-Named Entity).
\begin{lstlisting}
    
Output Snippet(ascending order)
  (0) word=I'm --> O: 1.380365
  (0) word=london --> B: 1.426504
  (0) word=JFK --> B: 1.468127
  (0) word=A --> O: 1.553408
  (0) digits=0 --> O: 1.581296
  (0) word=Chicago --> B: 1.611910
  (0) word=facebook --> B: 1.886708
  (0) word=Pope --> B: 2.014509
  (0) word=YouTube --> B: 2.151073
  (0) word=Twitter --> B: 2.206732
  (0) word=RT --> O: 2.225394
  (0) cap=0 --> O: 2.264817
  (0) word=I --> O: 2.935141
  (0) word=twitter --> B: 3.594446
\end{lstlisting}

\newline
\newline
\item False Negative: Daily Mail - Chunking would help identify named entities that span multiple words.
False Positive: Prison - Check if a word is in the dictionary. If it is, it is more likely not to be a named entity.

\newline
\newline
\item Mistake:
I feel teamiphone(dev.txt) should be a named entity in context. It refers to a fandom rather than a literal team. 
I also think that hashtags should be considered. Useful NLP data could be mined from considering the hashtags as named entities, and they could be used to improve named entities. 
Mentions should be considered a named entity. @TimothyJMoore is a named entity, as they all refer to a specific twitter account. 
\end{enumerate}
\end{document}
