\documentclass[a4paper]{article}

\usepackage[english]{babel}
\usepackage[utf8]{inputenc}
\usepackage{amsmath}
\usepackage{graphicx}
\usepackage[colorinlistoftodos]{todonotes}

\title{Project Report - NER tagging for Twitter - CMPSCI 585}

\author{Apoorva Rao Balevalachilu and Armand Halbert}

\date{\today}

\begin{document}
\maketitle

\begin{abstract}

The task is to perform named entity recognition on tweets. We provide results in BIO notation. This report will consist of two sections: System building and Analysis.
\end{abstract}

\section{Introduction}

NER can be performed for the following tasks: 
\begin{itemize}
\item Question Answering
\item Textual Entailment
\item Coreference Resolution
\item Computational Semantics
\end{itemize}

An example of NER is given below: \\

"Germany’s representative to the European Union’s veterinary
committee Werner Zwingman said on Wednesday consumers should ..." \\
\\
Germany - B, 
European - B,
Union - I,
Werner - B, 
Zwingman - I,
All other tokens - O \\

Our system uses Conditional Random Fields to make predictions for NER. This model is discriminative. It does not assume that features are independent. The benefit of using a CRF is that while labeling it takes future observations into account. \\

We have used the starter code provided by Professor Brendan O' Connor and David Belanger, CRFSuite, NLTK, and performed some experiments with the Freebase API. \\

We achieved an F-score of 0.23644 before the end of the competition, but improved our system significantly after that. At present, our scores are: \\

\texttt{F = 0.3821,  Prec = 0.5761 (462/802),  Rec = 0.2859 (462/1616)} \\
\texttt{(3336 sentences, 46714 tokens, 1616 gold spans, 802 predicted spans)} \\


\section{System Building}

We created a feature extractor that produces the following types of features. 

% mention whether you found them helpful in the end

\subsection{Lexical or wordform features}
\begin{enumerate}
\item Lowercased version of the word
\item Is in upper case?
\item Is a digit?
\item Is "retweet or RT"?
\item Is a URL?
\item Is an emoticon?
\item Is an apostrophe s?
\item Is a date?
\end{enumerate}

These were checked using regular expressions in the python code in \texttt{simple\_fe.py}

\subsection{Character Affixes}
\begin{enumerate}
\item Affixes consisting of first character
\item Affixes consisting of first two characters
\item Affixes consisting of first three characters
\item Suffixes consisting of last three characters
\item Suffixes consisting of last two characters
\item Suffixes consisting of last character
\item Begins with a hashtag i.e. \#?
\item Is a mention i.e. begins with \@?
\end{enumerate}

These were extracted using simple string manipulation code in python. 

\subsection{Shape Features}
\begin{enumerate}
\item Reduce uppercase characters - (r'[A-Z]+','A')
\item Reduce lowercase characters - (r'[a-z]+','a')
\item Reduce lowercase characters - (r'[0-9]+','0')
\item Reduce punctuations - (r'[\^A-Za-z0-9]+','\$')
\end{enumerate}

We created the shape features using regular expressions in python. 

\subsection{Positional Offset Features}
\begin{enumerate}
\item Previous word
\item Next word
\item Word before previous word
\item Word after next word
\end{enumerate}

\subsection{Major Extension - Freebase}

%Edit this section
% we can also use http://www.clips.uantwerpen.be/conll2003/ner/

\subsection{Major Extension - POS Taggers}

%Edit this section
%Tagger from NLTK and tagger from http://www.ark.cs.cmu.edu/TweetNLP/#pos

We use an external POS tagger to generate features.

\end{document}